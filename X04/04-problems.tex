\documentclass[
	ngerman,
	fontsize=10pt,
	parskip=half,
	titlepage=true,
	DIV=12
]{scrartcl}

\usepackage[utf8]{inputenc}
\usepackage{babel}
\usepackage[T1]	{fontenc}
\usepackage{lmodern}
\usepackage{microtype}
\usepackage{color}
\usepackage{csquotes}

\usepackage{hyperref}

\newcommand*{\tabcrlf}{\\ \hline}

\usepackage{amsmath}

\usepackage{minted}
	\usemintedstyle{friendly}

\newcommand*{\inPy}[1]{\mintinline{python}{#1}}

\newcommand*{\ie}{i.\,e. }
\newcommand*{\eg}{e.\,g. }
	
\begin{document}

\part*{Python Problems 04, Winter 2021/22}
\section{Dot Product (1\;P)}
Compute the dot product of two vectors $\vec{a} \cdot \vec{b}$ by expanding on the following code.

Reminder: the dot product is defined as:
\begin{align*}
	\vec{a} \cdot \vec{b} &= \sum_{j} a_j \cdot b_j
\end{align*}
That is, you multiply component-wise and sum up the products.

Start from this code:
\begin{minted}{python}
vector_a = [ 3, 2, 1, 5, 7, 2, -1]
vector_b = [-7, 3, 7, 5, 6, 8,  1]

# Your code here

print(dotProduct)
\end{minted}

In other words: comute $3 \cdot (-7) + 2 \cdot 3 + 1 \cdot 7 + \ldots$

\emph{Hint:} The result is 74.

\textbf{If you don't know the sum notation ($\sum$):}\\
This is a short notation for sums with many summands. Instead of listing every single one, only a \emph{generic form} with a \emph{summation index} is written. This summation index is a sequential number, identifying each summand.

Example:
\[\sum_{i=1}^4 i = 1 + 2 + 3 + 4 = 10 \]
The summation index here is $i$. It is always written underneath the \emph{summation symbol} $\sum$. Usually, below and above the $\sum$, there are also the \emph{lower and upper summation boundary}, \ie here $1$ and $4$. Behind the summation symbol there's an expression that describes all summands. The values $1 ... 4$ are plugged in for $i$ in sequence, thus forming all summands.

The summand can, of course, be more than a single number. In fact, it can be an arbitrary expression. Often, summands are \emph{elements of a list}. The sum of all elements in a list $a$ with four elements can thus be written as:
\[ \sum_{i=1}^4 a_i = a_1 + a_2 + a_3 + a_4 \]

Often, the lower and upper summation boundary are omitted. If this is done, it is implied to sum \emph{over all reasonable values of the summation index}. In the case of lists, this means to sum over all list elements.


\section{Counting Characters (3\;P)}
Create a string variable and put some text in it. Determine, how often each character appears in the string.

\emph{Example:}\\
The String \texttt{this is an ex-parrot} contains:
\begin{center}
\begin{tabular}{cc|cc}
(whitespace) & 3x & o            & 1x \\
-            & 1x & p            & 1x \\
a            & 2x & r            & 2x \\
e            & 1x & s            & 2x \\
h            & 1x & t            & 2x \\
i            & 2x & x            & 1x \\
n            & 1x 
\end{tabular}
\end{center}

\emph{Hint}: Use a \inPy{set} and a \inPy{dict}.


\section{Christmas Tree (2\;P)}
Print the following \enquote{Christmas tree} by putting the required number of whitespaces and asterisks on the screen:
\begin{minted}{text}
      *
     ***
    *****
   *******
  *********
 ***********
*************
\end{minted}

Writing seven \inPy{print} commands will not be counted as a solution. Instead, solve this problem with \inPy{for} loops. Keep your solution flexible by making it dependent on a single variable:
\mintinline{python}{height = 7}


\section{Probability (3\;P)}
Use Python to find an \emph{approximate} solution for the following question:
\begin{center}
	\emph{You break a rod into three pieces of arbitrary length. How big is the probability that from these three pieces a triangle can be formed?}
\end{center}

For this, follow these instructions:
\begin{itemize}
\item If you do an experiment $N$ times, and a certain outcome occurs $n$ times, then (for big $N$) $\frac{n}{N}$ is a reasonable approximation for the probability of
	the outcome.
\item Determine three random values $a, b, c$ that sum up to a fixed given lenght (\eg $a+b+c=1$). To do so, use these lines:
	\begin{minted}{python}
	import random
	
	lower = 0
	upper = 1
	print(random.uniform(lower, upper))
	\end{minted}
\item Now find out, if it is possible to form a triangle with sides $a, b, c$.
\item Repeat this process in a \inPy{for} loop many times over (at least $N=1000$ times). Count how often it was possible to form a triangle ($\rightarrow n$).
\end{itemize}
\end{document}