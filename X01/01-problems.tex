\documentclass[
	english,
	fontsize=10pt,
	parskip=half,
	titlepage=true,
	DIV=12
]{scrartcl}

\usepackage[utf8]{inputenc}
\usepackage{babel}
\usepackage[T1]	{fontenc}
\usepackage{lmodern}
\usepackage{microtype}
\usepackage{color}
\usepackage{csquotes}

\usepackage{hyperref}

\newcommand*{\tabcrlf}{\\ \hline}

\usepackage{minted}
	\usemintedstyle{friendly}

\newcommand*{\inPy}[1]{\mintinline{python}{#1}}
\newcommand*{\ie}{i.\,e. }
\newcommand*{\eg}{e.\,g. }

\begin{document}

\part*{Python Problems 01, Winter 2021}

\section{Setup and Testing of the Working Environment (1 P)}
Make sure you have correctly installed your Python interpreter and that you have an apt editor.

\textbf{For Windows-Users}:\\
If not done already, follow the steps up to section 1.4 in the tutorial you can find on GRIPS. If you have any questions, don't hesitate to ask your tutor.

\textbf{For Linux-Users}:\\
A Python interpreter should be installed by default installed. Open a terminal and enter
\begin{center}
	\texttt{python3}
\end{center}
If you get an error message, install the package \texttt{python3}. To do so, enter in the terminal window:
\begin{center}
	\texttt{sudo apt install software-properties-common}\\
	\texttt{sudo add-apt-repository ppa:deadsnakes/ppa}\\
	\texttt{sudo apt update}\\
	\texttt{sudo apt install python3}
\end{center}

Follow the instructions on \url{https://docs.python-guide.org/starting/install3/linux/}

Once you are able to start an interpreter session, type the following line:
\begin{center}
	\texttt{print("Hello World!")}
\end{center}
If the \enquote{reply} \texttt{Hello World!} appears on the screen, your interpreter works as intendet.

End the Python interpreter session. To do so, type \texttt{quit()}. Now download the file  \texttt{Install-Test-py.py} from GRIPS. Navigate to the directory where you saved the file ($\rightarrow$ \texttt{cd}), then enter:
\begin{center}
	\texttt{python3 Install-Test-py.py}
\end{center}

If this triggers an error message, perform the following steps:
\begin{itemize}
\item Install the Python package manager pip. For this, enter in the terminal:\\
	\texttt{sudo apt install python3-pip}
\item Install the packages numpy, scipy, matplotlib und pandas. For this, type into the terminal:\\
	\texttt{python -m pip install --user numpy scipy matplotlib pandas}
\end{itemize}
This may take several minutes to complete.

If everything works fine, running \texttt{Install-Test-py.py} prints multiple success messages in the terminal and opens a window showing a sine curve.

You can now start to work the exercise sheets. For your convenience, I recommend installing the code editor spyder. You can do so using the usual package sources. Type into the terminal:
\begin{center}
\texttt{sudo apt install spyder}
\end{center}

If you don't want to use Spyder, good alternative/lightweight editors are kate, geany or gedit.

\textbf{For Mac-Users:}\\
Sorry, I have no Mac and cannot test... but following the Tutorial for Windows Users seems to work well. Ask your tutor if you have any questions.


\section{Data types (1 P)}
Interpret the following code:
\begin{minted}[linenos]{python}
i = 2
f = 2.0
s = "2"

print(i + i)
print(i + f)
print(i * s)

#print(f * s)
\end{minted}

Which data type is associated with the expression \inPy{i + i}?\\
Which data type is associated with the expression \inPy{i + f}?\\
Which data type is associated with the expression \inPy{i * s}?\\
Why would \inPy{print(f * s)} trigger an error message?

\section{Complex Numbers (1 P)}
Create a variable \texttt{var} that stores the imaginary unit $i$. Convince yourself that Python evaluates $i^2$ as $-1$. What is the meaning of \texttt{var.real} and \texttt{var.imag}? Test with other complex numbers to confirm your assumption.

\section{Strings (1 P)}
Create a variable storing the following string:
\begin{center}
	\emph{"That's a pity", she said.}
\end{center}

Test this by printing the variable's content with \inPy{print}.
%s = '"That' + "'s a pity" +'", she said.'

\section{Table (2 P)}
Imagine you are working with a table of width $w$ and height $h$. The cells in this table are enumerated beginning with 0. So, for $w = 3, h = 4$ you would get these cell IDs:
\begin{center}
\begin{tabular}{|c|c|c|}
	\hline
	 0 &  1 &  2 \tabcrlf 
	 3 &  4 &  5 \tabcrlf
	 6 &  7 &  8 \tabcrlf
	 9 & 10 & 11 \tabcrlf
\end{tabular}
\end{center}

Write a program that computes the number of row and column when given a cell ID, and prints it on screen. Use a variable \texttt{cellID} in which you prepare the cell ID.

Example: If \inPy{cellID = 5} holds, your code should output \texttt{row 2, column 3}.

\end{document}

