\documentclass[
	ngerman,
	fontsize=10pt,
	parskip=half,
	titlepage=true,
	DIV=12
]{scrartcl}

\usepackage[utf8]{inputenc}
\usepackage{babel}
\usepackage[T1]	{fontenc}
\usepackage{lmodern}
\usepackage{microtype}
\usepackage{color}
\usepackage{csquotes}

\usepackage{hyperref}

\newcommand*{\tabcrlf}{\\ \hline}

\usepackage{amsmath}

\usepackage{minted}
	\usemintedstyle{friendly}

\newcommand*{\inPy}[1]{\mintinline{python3}{#1}}
\newcommand*{\ie}{i.\;e. }
\newcommand*{\eg}{e.\;g. }

\newcommand{\thus}{\ensuremath{\rightarrow}}

	
\begin{document}

\part*{Python Problems 07, Winter 2021/22}

\section{Chain Generator (1 P)}
Write a function  \texttt{chainGenerator} that takes an arbitrary number of arguments. The function should return a string that comprises of these elements, separated by a freely chooseable string \texttt{sep}. If no such string is indicated, by default a hypen (\texttt{-}) should be used. Your programm should work like this:
\begin{minted}{python}
print( chainGenerator(1, 2, 3) )                  # output: 1-2-3
print( chainGenerator(1, 2, 3, 4, sep="~~~") )    # output: 1~~~2~~~3~~~4
\end{minted}

\emph{Hint: You may need \inPy{str} to convert numbers to strings.}


\section{Sort by vector length (1 P)}
Let there be a list of vectors:
\mint{python}{data = [(7, 3), (5, 5), (8, 2), (9, 1), (6, 4)]}

Sort this list by length of the vectors.

\emph{Hint}:\\
A vector $(x, y)$ has length $\sqrt{x^2 + y^2}$. You can solve this problem with a single line of code.


\section{Binary Search (2 P)}
Write a \emph{recursive} function that searches a \emph{sorted} list, and tells whether or not a given element is in the list (\ie it should return either \inPy{True} or \inPy{False}). In other words, your function should emulate the behaviour of the \inPy{in} operator:
\begin{minted}{python}
data = [1, 5, 6, 7, 8, 42, 96, 666, 1337, 2112]
searchTerm = 42
result = searchTerm in data
\end{minted}

An \emph{iterative} solution could look like this:
\begin{minted}[linenos]{python}
data = [1, 5, 6, 7, 8, 42, 96, 1337, 2112]
searchTerm = 42

def iterativeSearch(searchterm, data) :
    for element in data :
        if element == searchTerm :
            return True
    else :
        return False

result = iterativeSearch(searchTerm, data)
\end{minted}

We now want to make use of the fact that we are given a \emph{sorted} list. With this, we can implement an algorithm that works much faster. We consider the following:
\begin{itemize}
\item We can compare \texttt{searchTerm} with the element \emph{in the middle of our list} \texttt{data}.
\item If the mid element is \emph{smaller} than \texttt{searchTerm}, it is sufficient to search the \emph{second half} of the list
\item Otherwise, it is sufficient to search the \emph{first half} of the list
\item It is trivial to decide whether or not \texttt{searchTerm} is in a list with only one element.
\item If we have a non-trivial case, we can recursively apply our first thoughts to analyze sub-lists.
\end{itemize}

Now write a function that follows this train of thought to find out whether or not \texttt{searchTerme} is in a list \texttt{data}.

How many comparisons does this algorithm need on average when searching a list of length $N$? How many comparisons does \texttt{iterativeSearch} need on average?


\section{Integral (II) (3 P)}
Write a function that takes a function as an argument, and that \emph{returns a function}. The returned function should compute an approximation for the antiderivative\footnote{German: Stammfunktion} with lower integration bound 0, \ie it should compute an approximation to:
\[ \int_0^x f(x) \;\text{d}{x} \]

You should be able to pass an optional argument \texttt{N} to set the accuracy of the approximation.
In other words, your function \texttt{antiderivative} should be usable in the following way:

\begin{minted}[linenos]{python3}
import math

def antiderivative (f, N = 1000)
#     your code here

F = antiderivative( math.sin )

print( F(math.pi) )    # should output approximately 2.0
\end{minted}

\emph{Hint}:\\
You can use most of your code from sheet 6, task 3.\\
Take inspiration from the lecture slides when we discussed the numerical derivative.
\end{document}
