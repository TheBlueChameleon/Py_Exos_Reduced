\documentclass[
	ngerman,
	fontsize=10pt,
	parskip=half,
	titlepage=true,
	DIV=12
]{scrartcl}

\usepackage[utf8]{inputenc}
\usepackage{babel}
\usepackage[T1]	{fontenc}
\usepackage{lmodern}
\usepackage{microtype}
\usepackage{color}
\usepackage{csquotes}

\usepackage{hyperref}

\newcommand*{\tabcrlf}{\\ \hline}

\usepackage{amsmath}

\usepackage{minted}
	\usemintedstyle{friendly}

\newcommand*{\inPy}[1]{\mintinline{python}{#1}}

\newcommand*{\ie}{i.\,e. }
\newcommand*{\eg}{e.\,g. }
	
\begin{document}

\part*{Python Problems 03, Winter 2021}
\section{Exploring Modules and Classes (1 P)}
With \inPy{help} you can show a brief description of a command/method/... Type \texttt{help(command)} into the interpreter section of your IDE to see an explanation of how to use the \texttt{command}. For example, \inPy{help(print)} will show you:

\begin{minted}{text}
Help on built-in function print in module builtins:

print(...)
    print(value, ..., sep=' ', end='\n', file=sys.stdout, flush=False)
    
    Prints the values to a stream, or to sys.stdout by default.
    Optional keyword arguments:
    file:  a file-like object (stream); defaults to the current sys.stdout.
    sep:   string inserted between values, default a space.
    end:   string appended after the last value, default a newline.
    flush: whether to forcibly flush the stream.
\end{minted}

The command \inPy{dir} lists the methods that can be used with an object. For example, you can type \inPy{dir(int)} into the interpreter section of your IDE to see a list of all methods that can be applied to an \inPy{int}.


Start from the following code:
\begin{minted}[linenos]{python}
import math

myList = ["Dusky", "Joe", "Hartington"]

print("### ELEMENTS IN MODULE MATH:")
print(dir(math))
print()

print("### ELEMENTS IN CLASS LIST:")
print(dir(myList))
\end{minted}

Using \inPy{help}, try to find out what \inPy{math.factorial}, \inPy{math.radians}, \inPy{myList.index}, \inPy{myList.pop} and \inPy{myList.clear} are good for. Expand on the above code to validate your assumptions.


\section{Substrings (1 P)}
Start from the string variable:
\mint{python}{stringVar = "My Hovercraft is full of eels"}
From this, create a string that contains only every second character of \texttt{stringVar}, beginning with the second character.

Create a new \inPy{list} variable where each list entry is one word in \texttt{stringVar}.

\emph{Hint}: Loop up \texttt{help(str.split)}.


\section{Matrix (1 P)}
Create \emph{one single variable} \inPy{matrix} that contains the following information:
\begin{equation*}
	\texttt{matrix} = \begin{pmatrix}
		1 & 2 & 3 \\
		4 & 5 & 6 \\
		7 & 8 & 9
	\end{pmatrix}
\end{equation*}

Use exactly three \inPy{print} commands to put a representation of this matrix on screen!

How can you change a single number in the matrix?

\section{Telefone book (1 P)}
Create \emph{one single list} that holds name, age and height of different people, \eg :
\begin{center}
	\begin{tabular}{ccc}
	Peter  & 19 & 1.80 \\
	Jasmin & 20 & 1.65 \\
	Alex   & 22 & 1.93
	\end{tabular}
\end{center}

Apply the method \inPy{sort} on this list. Given your current knowledge: how can you make Python sort the list according to another criterion (\eg sort by age)?

\section{Alternating List (1 P)}
Create the following output on screen:

\begin{minted}{text}
['a', 'b', 'a', 'b', 'a', 'b', 'a', 'b', 'a', 'b', 'a', 'b', 'a', 'b', 'a', 'b', 'a', 'b']
\end{minted}

The code creating this output should be shorter than the text generated.

\emph{Hint:} There are exactly 9 \texttt{a}s and \texttt{b}s.


\section{Erroneous Matrix Access (1 P)}
Regard the following code:

\begin{minted}[linenos]{python}
A = [ [1,2,3] ] * 3
print(A)
A[0][2] = 5
print(A)
\end{minted}

This creates a $3 \times 3$ matrix. Why does line 3 of this code alter more than one matrix element?


\section*{Optional Bonus Problem: Nesting (3 P)}
Explain the behaviour of the following code:

\begin{minted}[linenos]{python}
A = []
A.append(A)

print(A)
print(len(A))
print(len(A[0]))
print(len(A[0][0]))

A[0][0] = 1
print(A)
\end{minted}

What would be the memory model of this list?
\end{document}