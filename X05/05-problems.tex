\documentclass[
	english,
	fontsize=10pt,
	parskip=half,
	titlepage=true,
	DIV=12
]{scrartcl}

\usepackage[utf8]{inputenc}
\usepackage{babel}
\usepackage[T1]	{fontenc}
\usepackage{lmodern}
\usepackage{microtype}
\usepackage{color}
\usepackage{csquotes}
\usepackage{tabto}

\usepackage{hyperref}

\usepackage{graphicx}
\usepackage{wrapfig}
\usepackage[bf]{caption}
	\captionsetup{format=plain}

\newcommand*{\tabcrlf}{\\ \hline}

\usepackage{amsmath}

\usepackage{minted}
	\usemintedstyle{friendly}

\newcommand*{\inPy}[1]{\mintinline{python3}{#1}}
\newcommand*{\ie}{i.\;e. }
\newcommand*{\eg}{e.\;g. }

\newcommand{\thus}{\ensuremath{\rightarrow}}

\begin{document}

\part*{Python Problems 05, Winter 2021/22}
\section{Cross Sum (1\;P)}
Write code that takes an integer as input and computes the cross sum of this number.


\section{Fizzbuzz (1 + 1\;P)}
In the Game \emph{FizzBuzz}, players count from one upwards. If a number is divisible by three, the player should not name the number, but say \emph{fizz}. If the number is divisible by five, players should say \emph{buzz}. If the number is both, divisible by three and five, players say \emph{fizzbuzz}. Hence, players count like this:
\begin{center}
	1, 2, fizz, 4, buzz, fizz, 7, 8, fizz, buzz, 11, fizz, 13, 14, fizzbuzz ...
\end{center}

Create a program that puts this sequence on the screen.

\emph{Optionally: (+1 P)}\\
Write your code such that the rules can be expanded easily. Make it such that you can easily change the divisors and the replacement words. It is possible to write your code such that adding a single line adds, for example, the rule: \emph{numbers divisible by seven are replaced by \enquote{tezz}}.


\section{Text To List (1\;P)}
Write a program that takes a string of numbers separated by commas and that computes a \inPy{list} containing the numbers from this.

\emph{Example}:\\
Input: \inPy{"1, 5, 99, -3, 5.7"} \tab
Output: \inPy{[1, 5, 99, -3, 5.7]}

\emph{Hint:} This can be done with a single line of code.


\section{Line Breaks (3\;P)}
Create a text variable that stores one line of long text. Create another \inPy{int} variable \texttt{width} that specifies how many characters fit in one line. With these two inputs, write an algorithm that prints out your text variable as several lines. None of these printed lines should be longer than \texttt{width} characters, and no words should be split.

Example:\\
The input
\begin{minted}{python3}
text = 'We shall say "Ni" again to you, if you do not appease us.'
width = 20
\end{minted}

should produce the output:
\begin{minted}{text}
We shall say "Ni" 
again to you, if you 
do not appease us.
\end{minted}



\section{Prime Number Sieve (2\;P)}
Computing quotients (and hence also the remainder of a division, \ie the modulus) is a relatively costly operation: it takes about ten times longer than an addition (on usual processor architectures). We want to avoid divisions when solving time critical problems.

The \emph{Sieve of Eratosthenes} (named after Eratosthenes of Kyrene, born in ancient Greece, between 276 and 273 BCE) is an algorithm for finding all prime numbers between 2 and some arbitrary upper boundary $N$ that works \emph{without division/modulus}.

Write Python code that computes a \inPy{list} of prime numbers using Eratosthenes sieve. To do so, create a \inPy{list} of length $N+1$ and fill it with all integers between $0$ and $N$. Then, eliminate all multiples of 2 from your list (\eg by setting them to 0). Continue by eliminating all multiples of 3, and so on, unless the number in question has already been eliminated. For example, you need not eliminate all multiples of four, since four has been removed when eliminating all multiples of 2. Skip the fours and go on to the fives, and so on.
\end{document}
