\documentclass[
	ngerman,
	fontsize=10pt,
	parskip=half,
	titlepage=true,
	DIV=12
]{scrartcl}

\usepackage[utf8]{inputenc}
\usepackage{babel}
\usepackage[T1]	{fontenc}
\usepackage{lmodern}
\usepackage{microtype}
\usepackage{color}
\usepackage{csquotes}

\usepackage{hyperref}


\usepackage{graphicx}
\usepackage{wrapfig}
\usepackage[bf]{caption}
	\captionsetup{format=plain}

\newcommand*{\tabcrlf}{\\ \hline}

\usepackage{amsmath}

\usepackage{minted}
	\usemintedstyle{friendly}

\newcommand*{\inPy}[1]{\mintinline{python3}{#1}}
\newcommand*{\ie}{i.\;e. }
\newcommand*{\eg}{e.\;g. }

\newcommand{\thus}{\ensuremath{\rightarrow}}

	
\begin{document}

\part*{Python Problems 11, Winter 2021/22}
\section{Blog Database (2\;P)}
On GRIPS, you'll find the file \texttt{11-base.py}, which contains this code:
\begin{minted}[linenos]{python3}
def submit_post (ID, blog, **kwargs) :
    pass

blog_posts = [
    {'ID' : 'af853d12', 'Photos': 3, 'Likes': 21, 'Comments': 2},
    {'ID' : 'af853e09', 'Likes': 13, 'Comments': 2, 'Shares': 1},
    {'ID' : 'af853e22', 'Photos': 5, 'Likes': 33, 'Comments': 8, 'Shares': 3},
    {'ID' : 'af853f00', 'Comments': 4, 'Shares': 2},
    {'ID' : 'af853fa3', 'Photos': 8, 'Comments': 1, 'Shares': 1},
    {'ID' : 'af85402b', 'Photos': 3, 'Likes': 19, 'Comments': 3}
]

total_likes = 0

for post in blog_posts:
    total_likes = total_likes + post['Likes']

print(f"Total likes: {total_likes}")

submit_post('af85402b', blog_posts,
            Title = 'Yoda was wrong: There is a try!')
submit_post('af85402c', blog_posts,
            Title = 'Dr. Pythonlove Or: How I Learned to Stop Worrying and Love the Exception')

print(blog_posts)
\end{minted}

As you can see, the code will fail to run. Insert a \inPy{try .. except} block that catches the blog posts where no \texttt{Likes} information is available. The block should automatically add the entry \inPy{'Likes' : 0} to the corresponding entry in \texttt{blog\_posts}.

Then, flesh out the function \texttt{submit\_post}. It should append a blog (given by the \texttt{kwargs}) to the list \texttt{blog}, but only if the ID was not used before. If the ID is already in the \texttt{blog}, a user defined \texttt{DuplicatePostError} should be raised.

That means, in the above example, lines 20 and 21 should raise a \texttt{DuplicatePostError}. Lines 22 and 23 should have the effect of adding the entry
\begin{minted}{python3}
{'ID': 'af85402c',
 'Title': 'Dr. Pythonlove Or: How I Learned to Stop Worrying and Love the Exception'}
\end{minted}
to the list \texttt{blog\_posts}.


\section{Non-Abortable Code (1\;P)}
On GRIPS, you'll find the file \texttt{11-base.py}, which contains this code:
\begin{minted}[linenos]{python3}
import time

print("Press CTRL + C to prevent waiting very long:")

for x in range(1000) :
    try :
        time.sleep(.01)
    except :
        print("\nNope, I fooled you!")
\end{minted}

If you run the code and try to abort it by pressing \texttt{CTRL + C}, only the line \texttt{Nope, I fooled you} will appear and you have to wait the full ten seconds for the code to finish.

Why is that so? What would you need to change to restore the capacity to abort the code execution?


\section{N-Fold Concatenation (2\;P)}
Write a function \texttt{concatenateNFold} that computes the value of a function $f$ after n-fold concatenation. That is, the result of 
\texttt{concatenateNFold(f, 3, x)}
should be $f(f(f(x)))$.

\emph{Hint}:\\
Test your code with $f(x) = x + 1$ and $x = 0$. If your code is correct, then the output for \texttt{concatenateNFold(f, N, 0)} should be $N$.


\section{Matchmaking (2+4\;P)}
Create a list \texttt{socks} which contains numbers. Each number should be \emph{exactly twice} in your list, but at a random position. For example, your list may look like this:
\mint{python}{socks = [0, 2, 0, 1, 1, 2]}

Now create a new list \texttt{matches}, that stores for each element in \texttt{socks}, where its counterpart can be found, \ie at which index the same number can be found again. For the example shown above, \texttt{matches} should look like this:
\mint{python}{matches = [2, 5, 0, 4, 3, 1]}

\emph{Hint:}\\
To create \texttt{socks}, look up \texttt{help(random.shuffle)}. Obviously, you will need \inPy{import random}.

\emph{Background Information / Bonus task}:\\
This task was conceived when sorting socks in the laundry. I discussed various methods of optimizing the process with my fellow lodger, who then wanted to implement it himself. Later on, I nerd-sniped\footnote{\url{https://xkcd.com/356/}} others with this, ending in Nils Meyer (one of my then tutors) coming up with the QuickSock-Algorithm, which you can find on GRIPS later on. QuickSock is a variant of \emph{QuickSort}\footnote{\url{https://en.wikipedia.org/wiki/Quicksort}}, a sorting algorithm discussed in the lecture \emph{Algorithmen und Datenstrukturen} each summer term.

Python's own sorting algorithm is called \emph{TimSort} and is based upon QuickSort. and \emph{Mergesort}\footnote{\url{https://en.wikipedia.org/wiki/Merge_sort}}. A \emph{Mergesort-like} recursive approach can be found in the solutions on GRIPS. You can attempt such a recursive approach for the bonus points, too.
\end{document}
